%% Date Created: March 4, 2022, 
%% Last Modified: October 16, 2024
%% Available on GitHub at: https://github.com/elcarrillo/academic_cv_latex

\documentclass[a4paper,10pt]{article}
% Package imports for layout, fonts, icons, colors, etc.
\usepackage[margin=2cm]{geometry} % Page layout
\usepackage{titlesec}             % Custom section titles
\usepackage{setspace}             % Line spacing
\usepackage{enumitem}             % Custom list formats
\usepackage{xcolor}               % Custom colors
\usepackage{fancyhdr}             % Custom headers/footers
\usepackage{hyperref}             % Clickable URLs and links
\usepackage{fontawesome}          % Social media icons (GitHub, email)
\usepackage{tabularx}             % Responsive tables
\usepackage{hanging}              % Hanging indents for citations
\usepackage{helvet}               % Sans-serif font (Helvetica)
% \usepackage{lastpage}           % allows for referencing last page, used in footer
\usepackage{datetime}             % Custom date/time

% redefine \today to show only month and year
\renewcommand{\today}{\monthname[\the\month] \the\year}


\renewcommand{\familydefault}{\sfdefault} % Set default font to sans-serif
\definecolor{NavyBlue}{RGB}{0,0,128}      % custom colors

% title formatting for sections
\titleformat{\section}  
  {\color{NavyBlue}\normalfont\Large\bfseries}{\thesection}{1em}{}[{\titlerule[0.8pt]}] 
\titleformat{\subsection}  
  {\normalfont\bfseries}{\thesubsection}{1em}{\underline}[]
\titleformat{\subsubsection}  
  {\normalfont\bfseries\itshape}{\thesubsubsection}{1em}{}[{}]

\titlespacing*{\subsubsection}{0pt}{0ex}{0ex}


% line spacing
\setstretch{1.2}

% Footer
\pagestyle{fancy}  % Set the general footer for all pages
\fancyhf{}         % Clear all header and footer fields

%% Define footer for all pages 
\renewcommand{\headrulewidth}{0pt}  % Remove the line at the top
\fancyfoot[L]{Edgar L. Carrillo}    % Left-aligned name
\fancyfoot[C]{\thepage}             % Center-aligned page number
\fancyfoot[R]{\today}               % Right-aligned date

% Custom footer for the first page
\fancypagestyle{firstpagefooter}{
    \fancyhf{}  % Clear header and footer on the first page
    % \fancyfoot[L]{Custom text for the first page}  % Custom left footer text
    \fancyfoot[C]{\thepage}  % Custom center footer text
    \fancyfoot[R]{\today}  % Custom right footer text
}
%% to use, insert the following command after title text: \thispagestyle{firstpagefooter} 

% custom hyperlink formatting 
\hypersetup{
    colorlinks=true,  % Keep links like URLs visible
    linkcolor=black,  % Link color
    urlcolor=blue,    % URLs color
    citecolor=black,  % Citation links colors
    linkbordercolor={1 1 1}, % Hides boxes around links (for page numbers)
}

%% Custom subheading - can be used in place of subsection heading 
\newcommand{\subheading}[1]{\vspace{0.5cm}\noindent\makebox[\textwidth][l]{\underline{\textbf{#1}}}\par\vspace{0.3cm}} 
    %makebox controls overfull underlines

\raggedbottom %% stops latex from stretching content to fill the page

%% Section General Format:
% \section*{New Section Title}
% \begin{tabularx}{\textwidth}{>{\raggedright\arraybackslash}p{2.5cm} X}
% Year & \textbf{Role}, Organization, Location \\
%      & Description of role or project.
% \end{tabularx}

%%%%%%%%%%%%%%%%%%%%%%%%%%%%%%%%%%%%%%%%%%%%%%%%%%%%%%%%%%%%%%%%%%%%%%%%%%%%%%%%%%%%%%%%%
\begin{document}



\title{\LARGE\bfseries Edgar L. Carrillo}
\author{}
\date{}

\maketitle \vspace{-2.0cm} %vspace -# gets rid of space from empty \author and \date

\begin{center}
    PhD Student $\cdot$ Department of Earth Science $\cdot$ University of Oregon \\
    
    \faEnvelope\ \href{mailto:elcar@uoregon.edu}{elcar@uoregon.edu} $\cdot$ 
    \faGlobe\ \href{https://elcarrillo.github.io/portfolio/}{elcarrillo.github.io} $\cdot$ 
    \faGithub\ \href{https://github.com/elcarrillo}{github.com/elcarrillo} %\cdot$ 
    % \faTwitter\ \href{https://x.com/EdgarLCarrillo}{@EdgarLCarrillo}
\end{center}

%% use custom footer on the first page
\thispagestyle{firstpagefooter}


\section*{Education}
\begin{tabularx}{\textwidth}{>{\raggedright\arraybackslash}p{2.5cm} X}
2024 - Present & \textbf{University of Oregon}, Eugene, OR \\
               & Ph.D. in Earth Science, concentration in Volcano Physics \\
               & Thesis: \textit{Dynamics of Volcanic Flows} \\
               & Advisor: Leif Karlstrom \\
\\
2021 - 2023    & \textbf{Fisk University}, Nashville, TN \\
               & M.S. in Physics, concentration in Volcano Physics \\
               & Thesis: \textit{Dynamics of Water-Rich Volcanic Plumes} \\
               & Advisor: Kristen Fauria, Vanderbilt University \\
\\
2016 - 2019    & \textbf{California State University}, San Bernardino, CA \\
               & B.S. in Physics, concentration in Applied Physics \\
               & Dean's List 2018 \\
\end{tabularx}


\section*{Publications}

\subsubsection*{In Review} %% using subsubsection heading here since its more subtle heading 
\hangindent=1.5em \hangafter=1 %% hangindent enviromnment gives citation style formatting
\textbf{Carrillo, E.L.}, Fauria, K.E., Mittal, T., Mastin, L.G., (2024). Effects of External Water on Explosive Eruption Plume Height.
%%% When using multiple line items with \hangitem in the same section, use \noindent to reset indentation for every new item

\subsubsection*{In Prep}
% \noindent
\hangindent=1.5em \hangafter=1
Ruefer, A.C., Kelly, L.J., Guilherme, G.A.R., \textbf{Carrillo, E.L.}, Hickernell, S., Ward, S., Winslow, H., Ruprecht, P., (2024). One small step in the crust, one giant leap for magma: Insights into magma differentiation from basalt to rhyolite at Cordón Caulle derived from rhyolite-MELTS simulations.


\section*{Invited Talks}
\begin{tabularx}{\textwidth}{>{\raggedright\arraybackslash}p{1cm} X}
2024 & \textbf{Science by the Slice, a STEM Seminar Series}, ``Boom! The Physics of Volcanic Eruptions.'' Lane Community College, Eugene, OR. \\
2022 & \textbf{Bridge Research Celebration Day}, ``External Water Influence in Explosive Eruption Plumes.'' Vanderbilt University, Nashville, TN. \\
2022 & \textbf{Fisk Research Symposium}, ``Dynamics of Shallow Submarine Eruptions'' Fisk University, Nashville, TN.
\end{tabularx}

\section*{Conference Abstracts}

    \textbf{Carrillo, E.}, Fauria K.E., Mittal, T., Mastin, L.G. (2024, October). Effects of External Water on Explosive 
    \\ \indent Eruption Plume Height. SACNAS NDiSTEM 2024, Phoenix, AZ.\\
     Fauria K.E., \textbf{Carrillo, E.}, Mittal, T., Mastin, L.G., Jutzeler M. (2024, August). Plume Heights in Water-Rich 
     \\ \indent Explosive Eruptions. Cascades24, Bend, OR.\\
     \textbf{Carrillo, E.}, Fauria K.E., Mittal, T. (2023, December). External Water Effects on Explosive Eruption Plume
     \\ \indent Height. AGU Fall Meeting 2023, San Francisco, CA.


\section*{Honors and Awards}
    \begin{tabularx}{\textwidth}{>{\raggedright\arraybackslash}p{1cm} X}
    2024 & \textbf{College of Arts and Sciences Fellowship}, University of Oregon \\
    2024 & \textbf{IGEN Academic Travel Grant}, National Science Foundation \\
    2024 & \textbf{Summer School on Mathematics of Geophysical Flows Travel Grant}, Max Planck Institute for Mathematics in the Sciences \\
    2024 & \textbf{Earth Sciences Department Recruitment Award}, University of Oregon \\
    2023 & \textbf{College of Arts \& Sciences Fellowship}, Vanderbilt University \\
    2023 & \textbf{Geometry and Analysis of Fluid Flows Workshop Travel Grant}, Vanderbilt University \\
    2022 & \textbf{Graduate Student Support Grant}, American Physical Society \\
    2021 & \textbf{Fisk-Vanderbilt Masters-to-PhD Bridge Fellowship}, Fisk University \\
    2019 & \textbf{Math Alliance Predoctoral Scholar}, Purdue University
\end{tabularx}


\section*{Research Interests:} %% using list in wrapped line for compactness
    Submarine Volcanism, Oceanography, Pyroclastic Density Currents, Conduit Dynamics, Magma Bodies, Igneous Petrology, Fluid Mechanics, Mathematical Modeling

\section*{Research Experience}
    \begin{tabularx}{\textwidth}{>{\raggedright\arraybackslash}p{2.5cm} X}
        2024 - Present & \textbf{Research Assistant}, University of Oregon, Eugene, OR \\
                       & Investigated flow localization in fissure systems. \\
                       & Used numerical models to analyze the dynamics of conduits. \\
        \\
        2021 - 2024    & \textbf{Research Assistant}, Vanderbilt University, Nashville, TN \\
                       & Investigated the impact of external water and mass eruption rate on plume height. \\
                       & Assisted in geochemical and thermal modeling using Rhyolite-MELTS. \\
        \\
        2019 (Summer)  & \textbf{Research Experience for Undergraduates}, Oregon State University, Corvallis, OR \\
                       & Participated in NSF and NSA-funded summer research on the reliability assessment of the Power Grid using probability mass functions. \\
                       & Developed algorithms in R to simulate and analyze data.
    \end{tabularx}

\section*{Professional Experience}
    \begin{tabularx}{\textwidth}{>{\raggedright\arraybackslash}p{2.5cm} X}
        2020 - 2021 & \textbf{Software QA (contract)}, Waymo, Mountain View, CA \\
                    & Expanded scripts, reports, and dashboards to enhance feature-level performance. \\
                    & Simulated autonomous vehicle behavior for assistance systems and vehicle dynamics analysis. \\
                    & Mentored new employees and beta-tested major toolsets, suggesting enhancements. \\
        \\
        2016        & \textbf{Intern/Financial Analyst}, California Institute of Technology, Pasadena, CA \\
                    & Organized and classified data for financial analysis. \\
                    & Developed automated reporting systems to streamline key metric processing. \\
                    & Created financial models to assist in decision-making.
    \end{tabularx}



\section*{Field Experience}

    \begin{tabularx}{\textwidth}{>{\raggedright\arraybackslash}p{1cm} X}
        2024 & \textbf{Instructor}, Tennessee, USA (2 days) \\
             & \textit{Richland Creek:} Taught students how to assess a stream site (water quality and flood risk). \\
             & \textit{Fort Negley:} Observed sedimentary rocks, interpreted geological processes, and searched for outcrop fossils. \\\\
        
        2023 & \textbf{Participant}, Oregon, USA (6 days) \\
             & \textit{Cascadian Subduction Zone:} Participated in a short course on sedimentary deposits and their relation to past earthquakes. \\
             & \textit{Cascades:} Observed lava deposits at Newberry Volcano to understand lava flow dynamics and transport. \\\\
        
        2022 & \textbf{Participant and Student}, Italy (Various Locations) (15 days) \\
             & Assisted in sampling fiamme from the Ora Caldera. \\
             & Observed surface geology of the Ivrea Zone for insights into Earth's crust and upper mantle. \\
             & Attended a short course at Stromboli, focusing on effusive processes through lava deposit observations. \\\\
        
        2022 & \textbf{Student}, Long Valley Caldera, CA, USA (5 days) \\
             & Studied deposits to understand surface processes, including lava flow, pyroclastic flow, and shallow intrusions.
    \end{tabularx}

\section*{Teaching and Outreach}
\subsection*{National Association of Geoscience Teachers}
% \subheading{National Association of Geoscience Teachers}
\begin{tabularx}{\textwidth}{>{\raggedright\arraybackslash}p{2.5cm} X}
2024 & \textbf{Mentor}, Geosciences Education \& Mentorship Support \\
     & Advised undergraduate students at the University of Tennessee's (Knoxville, TN) Earth Science Department on academic matters.
\end{tabularx}

\subsection*{Vanderbilt University, Nashville, TN}
\begin{tabularx}{\textwidth}{>{\raggedright\arraybackslash}p{2.5cm} X}
2024 (Spring) & \textbf{Instructor for Physical Geology Lab}, Department of Earth and Environmental Science \\
              & Delivered hands-on laboratory sessions for an introductory physical geology course. \\
              & Supervised and assisted students in geological experiments and fieldwork. \\
              & Fostered an inclusive environment, encouraging student participation and passion for geology. \\\\

2023 \& 2024 (Summer) & \textbf{Instructor for Computational Physics Bootcamp}, Department of Physics and Astronomy \\
                      & Designed and delivered an introduction to computational physics, imparting practical skills to graduate students. \\\\

2023 (Fall) & \textbf{Tutor for Advanced Physical Chemistry}, Department of Physics and Astronomy \\
            & Provided tutoring for a graduate-level Physical Chemistry course. Enhanced student performance through effective teaching methodologies. \\\\

2023 - Present & \textbf{Reviewer for Young Scientist Journal}, Center for Science Outreach \\
               & Provided constructive feedback to authors and collaborated with editors to promote science literacy.
\end{tabularx}

\subsection*{Fisk University, Nashville, TN}
\begin{tabularx}{\textwidth}{>{\raggedright\arraybackslash}p{2.5cm} X}
2022 - 2024 & \textbf{Peer Mentor}, Department of Life and Physical Science \\
            & Guided first-year graduate students in the Fisk-Vanderbilt Bridge program on academic matters. \\
            & Organized group sessions to promote peer interaction and collaboration. \\\\

2024 (Spring) & \textbf{Guest Speaker for Professional Development Seminar} \\
              & Delivered a lecture on Time and Task Management to graduate students. Conducted interactive Q\&A sessions. \\\\

2024 (Spring) & \textbf{Guest Speaker for Biochemistry and Molecular Biology Course} \\
              & Presented stress management techniques to undergraduate students.
\end{tabularx}

\subsection*{California Institute of Technology, Pasadena, CA}
\begin{tabularx}{\textwidth}{>{\raggedright\arraybackslash}p{2.5cm} X}
2016 & \textbf{Volunteer}, Center for Learning and Outreach \\
     & Designed and presented science projects for K-12 students. Developed science projects for teachers.
\end{tabularx}

\section*{Training and Professional Development}
\begin{tabularx}{\textwidth}{>{\raggedright\arraybackslash}p{1cm} X}
2024 & \textbf{Graduate Employee Day of Teaching}, University of Oregon; Eugene, Oregon, USA \\
2024 & \textbf{Summer School on Mathematics of Geophysical Flows}, Max Planck Institute for Mathematics in the Sciences; Leipzig, Germany \\
2024 & \textbf{Classrooms, Careers, \& Communities: Maximizing Your TA Experience}, GSA Southeastern Section Meeting; Asheville, NC, USA \\
2021 & \textbf{Professional Development Seminar: Academic Mentorship}, Fisk University, Nashville, TN, USA
\end{tabularx}

\section*{Professional Memberships and Affiliations} %% using list in wrapped line for compactness
 American Physical Society (APS), American Geophysical Union (AGU), Geological Society of America (GSA), Society for the Advancement of Chicanos/Hispanics and Native Americans in Science (SACNAS)

\end{document}

%%%%%%%%%%%%%%%%%%%%%%%%%%%%%%%%%%%%%%%%%%%%%%%%%%%%%%%%%%%%%%%%%%%%%%%%%%%%%%%%%%%%%%%%%

%%% sample using tabbing environment instead of tabularx
%%% note: lines do not wrap automatically, so you will need to insert line breaks manually
% \section*{Field Experience}
% \begin{tabbing}
%           \hspace*{2.5cm} \= \hspace*{6cm} \= \kill
% 2024  \> \textbf{Instructor}, Tennessee, USA (2 days) \\
%        \> \textit{Richland Creek:} Taught students how to assess a stream site (water quality and flood risk). \\
%        \> \textit{Fort Negley:} Observed sedimentary rocks and interpreted geological processes, \\
%        \> \hspace*{1cm} searched for fossils in outcrops.\\\\

%  2023  \>\textbf{Participant},  Oregon, USA (6 days)\\
%         \>\textit{Cascadian Subduction Zone:} Participated in a short course on sedimentary deposits and their\\ 
%         \>relation to past earthquakes. \\
%         \>\textit{Cascades:} Observed lava deposits at Newberry Volcano to understand lava flow dynamics and\\
%         \> \hspace*{1cm} transport.\\\\

%   2022  \> \textbf{Participant and Student}  Italy (Various Locations) (15 days)\\
%         \>Assisted in sampling fiamme from the Ora Caldera. \\
%         \>Observed surface geology of the Ivrea Zone for insights into Earth's crust and upper mantle. \\
%         \>Attended a short course at Stromboli, focusing on effusive processes through lava deposit\\
%         \> observations.\\\\

% 2022 \> \textbf{Student}, Long Valley Caldera, CA, USA (5 days)\\
%      \> Studied deposits to understand surface processes, including lava flow, pyroclastic flow, \\
%      \> and shallow intrusions.
% \end{tabbing}